\documentclass[a4paper,12pt]{report}

\usepackage{alltt, fancyvrb, url}
\usepackage{graphicx}
\usepackage[utf8]{inputenc}
\usepackage{float}
\usepackage{xcolor}
\usepackage{hyperref}

\usepackage[italian]{babel}

\usepackage[italian]{cleveref}

\title{OOP24 - RUNWARRIOR}
\author{
    Samuele Bianchedi, Riccardo Cornacchia\\
    Francesca Gatti, Giovanni Maria Rava}
\date{\today}
\begin{document}
\maketitle
\chapter{Analisi}
\section{Descrizione e requisiti}
Il gruppo si pone come obbiettivo quello di realizzare una reinterpretazione del famoso gioco 
Super Mario Bross del 1986. il gioco consiste in un personaggio principale, un cavaliere, che tramite 
l'input dell'utente si muove in una mappa 2D. L'obbiettivo del cavaliere è salvare una princessa tenuta
prigioniera da uno stregone, completando diversi livelli che lo condurranno al castello nel quale è 
prigioniera. Nel gioco sarà possibile, tramite un menù, selezionare di giocare con un altro personaggio, 
un mago, che ha lo stesso obbiettivo del cavaliere. All'interno del gioco, oltre a diversi ostacoli, sono presenti nemici che il 
cavaliere deve uccidere per ottenere potenziamenti quali un armatura e la spada.
\subsection*{Requisiti funzionali}
\begin{itemize}
    \item Il personaggio deve avanzare, indietreggiare e saltare all'interno della mappa. Deve gestire le collisioni con nemici e ostacoli.
    \item Il personaggio può ottenere due potenziamenti, un'armatura e una spada che lo aiuteranno nella sua avventura.
    \item Gestione di nemici ed ostacoli diversi in base alla mappa.
    \item Creazione di un sistema di punteggio. Il punteggio verrà mostrato al com
\end{itemize}
\newpage
\subsection*{Requisiti non funzionali}
\begin{itemize}
    \item Le mappe hanno difficoltà crescenti. 
    \item Gestione di salvataggio e restart.
    \item Implementazione di un secondo potenziamento, spada o bastone
\end{itemize}
\section{Modello del Dominio}
\end{document}